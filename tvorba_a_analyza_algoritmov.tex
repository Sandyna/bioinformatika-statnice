\chapter[Tvorba a analýza algoritmov]{Tvorba a analýza algoritmov}
\label{tvorba_a_analyza_algoritmov} % id kapitoly pre prikaz ref

\section{Analýza časovej zložitosti algoritmov} 

\subsection{Definícia časovej zložitosti} TODO
\subsection{O-notácia} TODO
\subsection{Odhad časovej zložitosti rekurzívnych algoritmov používajúcich metódu rozdeľuj a panuj} TODO

\section{Algoritmy pre triedenie}\chapter[Tvorba a analýza algoritmov]{Tvorba a analýza algoritmov}
\label{tvorba_a_analyza_algoritmov} % id kapitoly pre prikaz ref

\section{Analýza časovej zložitosti algoritmov} 

\subsection{Definícia časovej zložitosti} TODO
\subsection{O-notácia} TODO
\subsection{Odhad časovej zložitosti rekurzívnych algoritmov používajúcich metódu rozdeľuj a panuj} TODO

\section{Algoritmy pre triedenie}

	\subsection{Efektívne algoritmy triedenia porovnávaním} TODO
	\subsection{Triedenie v lineárnom čase} TODO
	\subsection{Dolný odhad časovej zložitosti každého triedenia porovnávaním} TODO

\section{Dátové štruktúry v poli}
	\subsection{Pole s dynamickou veľkosťou – vektor} TODO
	\subsection{Zásobník} TODO
	\subsection{Fronta} TODO
	\subsection{Binárna halda a implementácia prioritnej fronty pomocou nej} TODO

\section{Usporiadané dátové štruktúry}
	\subsection{Binárne vyhľadávacie stromy} TODO
	\subsection{Usporiadaná množina} TODO
	\subsection{Usporiadané asociatívne pole – slovník} TODO
	\subsection{Vyvažovanie binárnych stromov} TODO

\section{Hešovanie}

	\subsection{Kolízie a rôzne spôsoby ich riešenia} TODO
	\subsection{Narodeninový paradox} TODO
	\subsection{Množina} TODO
	\subsection{Asociatívne pole} TODO

\section{Základné grafové algoritmy}

	\subsection{Reprezentácie grafu v pamäti}

		\paragraph{Matica susednosti} - dvojrozmerné pole $n \times n$, v ktorom môžeme reprezentovať graf pomocou boolov pre neohodnotený, integerov pre ohodnotený graf. \\
		Výhody:
		\begin{itemize}
			\item Vieme priamo zistiť, či x susedí s y
			\item Prehľadná a jednoduchá na prácu
			\item Priama reprezentácia ohodnotených hrán
		\end{itemize}
		Nevýhody:
		\begin{itemize}
			\item Pri riedkych grafoch zaberá zbytočne veľa miesta
			\item Pomalé hľadanie susedov vrcholu (potrebujeme prejsť všetkých n pozícií)
			\item Nejde priamočiaro reprezentovať multigrafy (pre každé políčko si ale môžme pamätať zoznam hrán)
		\end{itemize}

		\paragraph{Zoznam susedov} - pre každý vrchol si pamätáme zoznam jeho susedov, napr v poli dĺžky n spájaných zoznamov alebo vo vektore vektorov. Vrcholy môžeme označovať integermi.
		\begin{itemize}
			\item Rýchly prístup ku všetkým hranám idúcim z vrcholu
			\item Zaberá menej miesta ako matica susedov pri riedkých grafoch
			\item Priamočiara reprezentácia multigrafu
		\end{itemize}
		Nevýhody:
		\begin{itemize}
			\item K hranám vieme pristupovať len sekvenčne
			\item V ohodnotených grafoch si musíme pamätať páry čísel
		\end{itemize}


		\paragraph{Objektová reprezentácia}
		V objekte reprezentujúcom vrchol si pamätáme referencie na jeho susedov. V princípe rovnaké ako zoznam susedov, ale s čitateľnejším kódom.

	\subsection{Prehľadávanie do hĺbky a do šírky}
		\paragraph{DFS}
			Rekurzívne prehľadáva graf, pamätáme si už navštívené vrcholy.
			Využitie: 
			\begin{itemize}
				\item Hľadanie mostov v grafoch.
				\item Hľadanie artikulácií v grafoch.
				\item Hľadanie silno súvislých komponentov v orientovaných grafoch
				\item Testovanie planarity grafov.
			\end{itemize}


		\paragraph{BFS}

	\subsection{Topologické triedenie} TODO

\section{Najkratšie cesty v grafe}
	Vedia pracovať len na ohodnotených grafoch s nezápornými cenami hrán. Neexistujúce cesty medzi 
	\subsection{Dijkstrov algoritmus} TODO
		Vypočíta najlacnejšiu cestu v grafe začínajúcu v konkrétnom vrchole $v_{0}$. Výstupom je pole cien ciest z $v_{0}$ do každého $v \in G$.
		Worst-case: $O(|V|^{2})$
	\subsection{Floydov-Warshallov algoritmus} TODO
		Vypočíta najlacnejšiu cestu z každého vrcholu do každého vrcholu. Výstupom je matica najnižších cien ciest z $v \in G$ do každého $v \in G$.
		Worst-case: $O(|V|^{3})$

\section{Najlacnejšia kostra grafu}

	\subsection{Algoritmus Union-FindSet} TODO

	\subsection{Kruskalov algoritmus} TODO

\section{Násobenie matíc}

	\subsection{Naivný algoritmus} TODO
	\subsection{Strassenov algoritmus} TODO
	\subsection{Efektívne umocňovanie matice} TODO
	\subsection{Tranzitívny uzáver grafu pomocou umocňovania matíc} TODO


\section{Dynamické programovanie}
	Bottom-up riešenie
	\subsection{Konkrétne príklady použitia} 
		\paragraph{0-1 knapsack} TODO
		\paragraph{Floyd-Warshall} TODO
		\paragraph{Problém násobenia reťazca matíc} TODO

	\subsection{Charakterizácia problémov riešiteľných dynamickým programovaním} TODO
	
	\subsection{Porovnanie iteratívneho prístupu a rekurzie s memoizáciou} TODO


\section{Ďalšie princípy tvorby efektívnych algoritmov}
	\subsection{Rozdeľuj a panuj}
	TODO
	\subsection{Pažravé algoritmy}
		Používajú sa na riešenie optimalizačných problémov. Globálne optimálne riešenie je vytvorené pomocou postupných lokálne optimálnych riešení. Obvykle sú to iteratívne algoritmy, v ktorých problémy redukujeme na podproblémy menšieho rozsahu, v dôsledku čoho sú rýchle.
		\paragraph{Dijkstrov algoritmus}
		\paragraph{Kruskalov algoritmus}
		\paragraph{Racionálny knapsack}
	\subsection{Princíp vyváženosti}
		Stretávame sa s delením väčšieho problému na menšie, prípadne štruktúr na menšie podštruktúry, často vieme zvýšiť efektívnosť algoritmu ich vyváženosťou.
		\paragraph{Príklad:} Ak quicksort vyberá náhodný pivot, v priemernom prípade dosahuje zložitosť $O(n log n)$, pri zlom výbere pivotu to však môže byť až $O(n^{2})$ Chceme teda pivotom deliť na rovnaké - vyvážené časti (vyberieme medián).\\
		\paragraph{Príklad:} BST má v priemere výšku $log n$, vyhľadávanie zložitosť $O(log n)$, ak však vetvy nie sú vyvážené, môže dosiahnuť zložitosť $O(n)$.
	

	\subsection{Voľba vhodnej dátovej štruktúry}
		\paragraph{Abstraktný dátový typ} - abstrakcia nad dátovými štruktúrami popisujúca operácie, ktoré sa budú vykonávať. Podľa najčastejších operácií zvolíme implementáciu s ktorou budú najefektívnejšie.\\

		\begin{table}[h]
			\centering
			\caption{\textbf{Slovník}}
			\label{my-label}
			\begin{tabular}{|l|l|l|l|}
			\hline
			\textbf{Implementácia} & \textbf{Member} & \textbf{Insert} & \textbf{Delete}  \\ \hline
			Pole                   & $O(n)$            & $O(1)$           & $O(n)$             \\ \hline
			Utriedené pole         & $O(log n)$        & $O(n)$            & $O(n)$             \\ \hline
			2-3 stromy             & $O(log n)$        & $O(log n)$        & $O(log n)$          \\ \hline
			Heš                    & $O(1)$/$O(n)$      & $O(1)$/$O(n)$       & $O(1)$/$O(n)$        \\ \hline
			\end{tabular}
		\end{table}

		\begin{table}[h]
			\centering
			\caption{\textbf{Prioritná fronta}}
			\label{my-label}
			\begin{tabular}{|l|l|l|l|}
			\hline
			\textbf{Implementácia} & \textbf{Insert} & \textbf{Min/top} & \textbf{Delete min/pop}   \\ \hline
			Pole                   & $O(1)$            & $O(n)$             & $O(n)$                       \\ \hline
			Utriedené pole         & $O(n)$            & $O(1)$             & $O(1)$                       \\ \hline
			Halda                  & $O(log n)$        & $O(1)$             & $O(log n)$                  \\ \hline
			2-3 stromy             & $O(log n)$        & $O(log n)$         & $O(log n)$                   \\ \hline
			\end{tabular}
		\end{table}


	\subsection{Efektívne algoritmy triedenia porovnávaním} TODO
	\subsection{Triedenie v lineárnom čase} TODO
	\subsection{Dolný odhad časovej zložitosti každého triedenia porovnávaním} TODO

\section{Dátové štruktúry v poli}
	\subsection{Pole s dynamickou veľkosťou – vektor} TODO
	\subsection{Zásobník} TODO
	\subsection{Fronta} TODO
	\subsection{Binárna halda a implementácia prioritnej fronty pomocou nej} TODO

\section{Usporiadané dátové štruktúry}
	\subsection{Binárne vyhľadávacie stromy} TODO
	\subsection{Usporiadaná množina} TODO
	\subsection{Usporiadané asociatívne pole – slovník} TODO
	\subsection{Vyvažovanie binárnych stromov} TODO

\section{Hešovanie}

	\subsection{Kolízie a rôzne spôsoby ich riešenia} TODO
	\subsection{Narodeninový paradox} TODO
	\subsection{Množina} TODO
	\subsection{Asociatívne pole} TODO

\section{Základné grafové algoritmy}

	\subsection{Reprezentácie grafu v pamäti}

		\paragraph{Matica susednosti} - dvojrozmerné pole $n \times n$, v ktorom môžeme reprezentovať graf pomocou boolov pre neohodnotený, integerov pre ohodnotený graf. \\
		Výhody:
		\begin{itemize}
			\item Vieme priamo zistiť, či x susedí s y
			\item Prehľadná a jednoduchá na prácu
			\item Priama reprezentácia ohodnotených hrán
		\end{itemize}
		Nevýhody:
		\begin{itemize}
			\item Pri riedkych grafoch zaberá zbytočne veľa miesta
			\item Pomalé hľadanie susedov vrcholu (potrebujeme prejsť všetkých n pozícií)
			\item Nejde priamočiaro reprezentovať multigrafy (pre každé políčko si ale môžme pamätať zoznam hrán)
		\end{itemize}

		\paragraph{Zoznam susedov} - pre každý vrchol si pamätáme zoznam jeho susedov, napr v poli dĺžky n spájaných zoznamov alebo vo vektore vektorov. Vrcholy môžeme označovať integermi.
		\begin{itemize}
			\item Rýchly prístup ku všetkým hranám idúcim z vrcholu
			\item Zaberá menej miesta ako matica susedov pri riedkých grafoch
			\item Priamočiara reprezentácia multigrafu
		\end{itemize}
		Nevýhody:
		\begin{itemize}
			\item K hranám vieme pristupovať len sekvenčne
			\item V ohodnotených grafoch si musíme pamätať páry čísel
		\end{itemize}


		\paragraph{Objektová reprezentácia}
		V objekte reprezentujúcom vrchol si pamätáme referencie na jeho susedov. V princípe rovnaké ako zoznam susedov, ale s čitateľnejším kódom.

	\subsection{Prehľadávanie do hĺbky a do šírky}
		\paragraph{DFS}
			Rekurzívne prehľadáva graf, pamätáme si už navštívené vrcholy.
			Využitie: 
			\begin{itemize}
				\item Hľadanie mostov v grafoch.
				\item Hľadanie artikulácií v grafoch.
				\item Hľadanie silno súvislých komponentov v orientovaných grafoch
				\item Testovanie planarity grafov.
			\end{itemize}


		\paragraph{BFS}

	\subsection{Topologické triedenie} TODO

\section{Najkratšie cesty v grafe}
	Vedia pracovať len na ohodnotených grafoch s nezápornými cenami hrán. Neexistujúce cesty medzi 
	\subsection{Dijkstrov algoritmus} TODO
		Vypočíta najlacnejšiu cestu v grafe začínajúcu v konkrétnom vrchole $v_{0}$. Výstupom je pole cien ciest z $v_{0}$ do každého $v \in G$.
		Worst-case: $O(|V|^{2})$
	\subsection{Floydov-Warshallov algoritmus} TODO
		Vypočíta najlacnejšiu cestu z každého vrcholu do každého vrcholu. Výstupom je matica najnižších cien ciest z $v \in G$ do každého $v \in G$.
		Worst-case: $O(|V|^{3})$

\section{Najlacnejšia kostra grafu}

	\subsection{Algoritmus Union-FindSet} TODO

	\subsection{Kruskalov algoritmus} TODO

\section{Násobenie matíc}

	\subsection{Naivný algoritmus} TODO
	\subsection{Strassenov algoritmus} TODO
	\subsection{Efektívne umocňovanie matice} TODO
	\subsection{Tranzitívny uzáver grafu pomocou umocňovania matíc} TODO


\section{Dynamické programovanie}
	Bottom-up riešenie
	\subsection{Konkrétne príklady použitia} 
		\paragraph{0-1 knapsack} TODO
		\paragraph{Floyd-Warshall} TODO
		\paragraph{Problém násobenia reťazca matíc} TODO

	\subsection{Charakterizácia problémov riešiteľných dynamickým programovaním} TODO
	
	\subsection{Porovnanie iteratívneho prístupu a rekurzie s memoizáciou} TODO


\section{Ďalšie princípy tvorby efektívnych algoritmov}
	\subsection{Rozdeľuj a panuj}
	TODO
	\subsection{Pažravé algoritmy}
		Používajú sa na riešenie optimalizačných problémov. Globálne optimálne riešenie je vytvorené pomocou postupných lokálne optimálnych riešení. Obvykle sú to iteratívne algoritmy, v ktorých problémy redukujeme na podproblémy menšieho rozsahu, v dôsledku čoho sú rýchle.
		\paragraph{Dijkstrov algoritmus}
		\paragraph{Kruskalov algoritmus}
		\paragraph{Racionálny knapsack}
	\subsection{Princíp vyváženosti}
		Stretávame sa s delením väčšieho problému na menšie, prípadne štruktúr na menšie podštruktúry, často vieme zvýšiť efektívnosť algoritmu ich vyváženosťou.
		\paragraph{Príklad:} Ak quicksort vyberá náhodný pivot, v priemernom prípade dosahuje zložitosť $O(n log n)$, pri zlom výbere pivotu to však môže byť až $O(n^{2})$ Chceme teda pivotom deliť na rovnaké - vyvážené časti (vyberieme medián).\\
		\paragraph{Príklad:} BST má v priemere výšku $log n$, vyhľadávanie zložitosť $O(log n)$, ak však vetvy nie sú vyvážené, môže dosiahnuť zložitosť $O(n)$.
	

	\subsection{Voľba vhodnej dátovej štruktúry}
		\paragraph{Abstraktný dátový typ} - abstrakcia nad dátovými štruktúrami popisujúca operácie, ktoré sa budú vykonávať. Podľa najčastejších operácií zvolíme implementáciu s ktorou budú najefektívnejšie.\\

		\begin{table}[h]
			\centering
			\caption{\textbf{Slovník}}
			\label{my-label}
			\begin{tabular}{|l|l|l|l|}
			\hline
			\textbf{Implementácia} & \textbf{Member} & \textbf{Insert} & \textbf{Delete}  \\ \hline
			Pole                   & $O(n)$            & $O(1)$           & $O(n)$             \\ \hline
			Utriedené pole         & $O(log n)$        & $O(n)$            & $O(n)$             \\ \hline
			2-3 stromy             & $O(log n)$        & $O(log n)$        & $O(log n)$          \\ \hline
			Heš                    & $O(1)$/$O(n)$      & $O(1)$/$O(n)$       & $O(1)$/$O(n)$        \\ \hline
			\end{tabular}
		\end{table}

		\begin{table}[h]
			\centering
			\caption{\textbf{Prioritná fronta}}
			\label{my-label}
			\begin{tabular}{|l|l|l|l|}
			\hline
			\textbf{Implementácia} & \textbf{Insert} & \textbf{Min/top} & \textbf{Delete min/pop}   \\ \hline
			Pole                   & $O(1)$            & $O(n)$             & $O(n)$                       \\ \hline
			Utriedené pole         & $O(n)$            & $O(1)$             & $O(1)$                       \\ \hline
			Halda                  & $O(log n)$        & $O(1)$             & $O(log n)$                  \\ \hline
			2-3 stromy             & $O(log n)$        & $O(log n)$         & $O(log n)$                   \\ \hline
			\end{tabular}
		\end{table}
