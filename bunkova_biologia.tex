%\newcommand\tab[1][1cm]{\hspace*{#1}}
\chapter[Bunková biológia]{Bunková biológia}
\label{bunkova_biologia} % id kapitoly pre prikaz ref

%1
\subsection{Vnútorná organizácia buniek a ich pôvod v evolúcii}
Status: DONE\\
Source: Prezentácia 1\\
\\
\subsection{História a kľúčové objavy bunkovej biológie}
Robert Hooke -- termín bunka, organizmy sú z buniek\\
Antonie van Leewenhoek -- mikroskop\\
\subsection{Bunková teória}
Schwann, Schleiden, Remak, Virchow\\
Pôvodné tri:\\
\tab Živé organizmy sú z jednej alebo viacerých buniek (dišputa -- vírusy)\\
\tab Bunky sú základné štruktúrne a funkčné jednotky živých organizmov\\
\tab Vznikajú len delením preexistujúcich buniek (Waaaait. Prvá bunka?)\\
Additional: \\
\tab Podobné chemické zloženie\\
\tab Chemický systém, kde prebieha premena energií a metabolické reakcie\\
\tab DNA je genetický materiál\\
\subsection{Porovnanie prokaryotických a eukaryotických buniek}
0.3 mikm -- 0.7 mm, 9 mikm -- 800 mikm\\
Prokaryotické\\
\tab Archaea, Bacteria\\
\tab Jadro (nucleoid) voľne v cytosole\\
\tab Bez membránových organel\\
\tab Cirkulárna DNA (cirkulárny chromozóm)\\
\tab Ribozómy\\
\tab Archaea má karboxyzómy, plynové vezikuly, etc.\\
Eukaryotické\\
\tab Eukarya\\
\tab Jadro má vlastnú membránu, nucleolus\\
\tab Membránové organely, napr. mitochondrie, golgiho aparát\\
\tab Viac vlákien DNA (viac chromozómov)\\
\tab Ribozómy\\
\subsection{Komplexná organizácia eukaryotickej bunky, význam intracelulárnej kompartmentalizácie a vnútrobunkový dialóg}
Bunková štruktúra -- Čokoľvek v bunke (ribozóm, deliace vretienko...)\\
Bunkový kompartment -- časť bunky oddelená membránou al. proteínom (cytosól, jadro)\\
Bunková organela -- funkčné časti bunky obklopené membránou (mitochondria, plastidy)\\
\\
jadro\\
mitochondrie, hydrogenozómy\\
plastidy (rastlinné bunky)\\
endoplazmatické retikulum\\
Golgiho aparát\\
lyzozómy, vakuoly\\
peroxizómy\\
cytosol\\
\subsection{Vznik buniek v evolúcii}
RNA(Genotyp + Fenotyp) $\rightarrow$ RNP(Genotyp + Fenotyp) $\rightarrow$ DNA(Genotyp ~ DNA + Fenotyp(Proteínový))\\
Darwin\\
\tab jeden spoločný predok\\
Woese\\
\tab viacero vetiev $\rightarrow$ tree of life, archaea, bacteria, eucarya\\
\tab RNA selfreplicating teória\\
Darwinovský prah (Darwinian Treshold) -- bod, pred ktorým speciácia nebola možná, kvôli horizontálnemu transferu génov\\
\subsection{Pôvod komplexnej (eukaryotickej) bunky}
Lynn Margulis\\
Endosymbiotická teória\\
Evolučná mozaika\\
Niektoré organely (mitochondrie, plastidy) vznikli vďaka endosymbióze. Resp. eukarya vznikli ako symbióza archaea a procarya. Jadrový genóm pochádza z archaea a bacteria...\\
Reduktívna fáza -- strata časti genómu, funkcií, transfer génov do jadra\\
Expanzívna fáza -- vznik nových génov, horizont. gén. transfer prokaryotických génov, konverzia endosymbionta na organelu exportujúcu ATP\\
Mitochondrie majú vlastný genóm\\
Vodíková hypotéza\\

%2
\section{Bunkové jadro: štruktúra a dynamika chromozómov}
Status: Not started\\
Source: Prezentácia 2\\

\subsection{Prokaryotické, eukaryotické a organelové chromozómy}

\subsection{DNA a proteínové komponenty chromozómov}

\subsection{Distribúcia chromozómov pri delení buniek}

\subsection{Objav úlohy DNA}

\subsection{Replikačné stratégie DNA}

\subsection{Experimenty Meselsona a Stahla}

\subsection{Semikonzervatívny mechanizmus syntézy DNA}

\subsection{Iniciácia, elongácia a terminácia replikácie (replikačné počiatky, replikačné bubliny}

\subsection{Okazakiho fragmenty, leading a lagging vlákno). Replizóm}

\subsection{Kľúčové enzýmy v replikácii: DNA polymerázy, primázy, ligázy, helikázy, topoizomerázy, ssb proteíny}

%3
\section{Mechanizmy opravy poškodenej DNA}

\subsection{Poškodenia chromozomálnej DNA}

\subsection{Fyzikálne, chemické a biologické mutagény}

\subsection{Príčiny vzniku spontánnych mutácií}

\subsection{Reparačné mechanizmy (fotoreaktivácia, bázová a nukleotidová excízna reparácia, rekombinačná oprava, SOS odpoveď)}

\subsection{Ochorenia spôsobené defektmi v oprave DNA. }

%4
\section{Transkripcia a úlohy RNA v bunke}

\subsection{Úloha RNA v interpretácii genetickej informácie}

\subsection{Typy RNA (mRNA, rRNA, tRNA, malé RNA)}

\subsection{Katalytické vlastnosti RNA}

\subsection{Svet RNA a evolúcia živých systémov}

\subsection{Transkripcia}

\subsection{Iniciácia, elongácia a terminácia transkripcie}

\subsection{RNA polymerázy}

\subsection{Transkripčné faktory. Porovnanie transkripcie v prokaryotoch a eukaryotoch.}

%5
\section{Syntéza a distribúcia proteínov v bunkách}

\subsection{Objav a vlastnosti genetického kódu}

\subsection{Štruktúra a vlastnosti tRNA}

\subsection{Štruktúra a funkcie ribozómov}

\subsection{Ribozomálne RNA a proteínové komponenty ribozómu}

\subsection{Základné etapy translácie (iniciácia, elongácia a terminácia)}

\subsection{Porovnanie prokaryotickej a eukaryotickej proteosyntézy}

\subsection{Inhibítory proteosyntézy}

\subsection{Vnútrobunková lokalizácia proteosyntézy}

\subsection{Distribúcia proteínov v bunke.}

%6
\section{Princípy kontroly expresie génov}

\subsection{Definície génu}

\subsection{Úrovne kontroly expresie génov}

\subsection{Operónový model}

\subsection{Pokusy Jacoba a Monoda}

\subsection{Negatívna a pozitívna kontrola expresie}

\subsection{Katabolická represia.}

\subsection{Atenuácia}

\subsection{Regulácia životného cyklu fága lambda}

\subsection{Porovnanie kontroly génovej expresie v prokaryotických a eukaryotických bunkách}

\subsection{Kontrola na úrovni transkripcie a posttranskripčné úpravy RNA}

\subsection{Kontrola na úrovni translácie a posttranslačné úpravy proteínov.}

%7
\section{Úloha biologických membrán v eukaryotickej bunke}

\subsection{Kompartmentalizácia bunky}

\subsection{Štruktúra a funkcie membrán}

\subsection{Transport cez membrány}

\subsection{Vektorové procesy viazané na membrány}

\subsection{Úloha membrán v prenose nervového signálu.}

%8
\section{Mitochondrie a chloroplasty}

\subsection{Ultraštruktúra a funkcie semiautonómnych organel}

\subsection{Špecifické úlohy membrán mitochondrií a chloroplastov}

\subsection{Organelové genómy}

\subsection{Oxidatívna fosforylácia.}

\subsection{Fotosyntéza--fotofosforylácia}

%9
\section{Endoplazmatické retikulum, Golgiho aparát}

\subsection{Štruktúra, funkcie, biogenéza a distribúcia}

\subsection{Hladké a drsné endoplazmatické retikulum, sarkoplazmatické retikulum}

\subsection{Vezikulárny transport}

\subsection{Úloha v distribúcii a transporte proteínov v eukaryotickej bunke.}

%10
\section{Vakuoly, lyzozómy a peroxizómy}

\subsection{Štruktúra, funkcie, biogenéza a distribúcia}

\subsection{Metabolizmus}

\subsection{Klinický význam lyzozómov a peroxizómov. }

%11
\section{Cytoskelet ako dynamická štruktúra}

\subsection{Komponenty cytoskeletu}

\subsection{Cytoskelet ako pohybový aparát: vezikulárny transport, bunková motilita a delenie buniek}

%12
\section{Od jednotlivých buniek k tkanivám a mnohobunkovým organizmom}

\subsection{Bunkové povrchy}

\subsection{Cytoplazmatická membrána a bunková stena}

\subsection{Extracelulárna matrix}

\subsection{Bunky v sociálnom kontexte.}

\subsection{Biofilmy}

\subsection{Bunky ako súčasť tkanív}

\subsection{Epitely a medzibunkové spojenia}

\subsection{Quorum sensing.}

\subsection{Medzibunková komunikácia a bunková smrť.}
