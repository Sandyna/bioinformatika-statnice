\chapter[Metódy v bioinformatike]{Metódy v bioinformatike}
\label{metody_v_bioinformatike.tex} % id kapitoly pre prikaz ref

\href{http://compbio.fmph.uniba.sk/vyuka/mbi/index.php/Predn%C3%A1%C5%A1ky_a_pozn%C3%A1mky}{Link na skriptá}

\chapter{Sekvenovanie, zostavovanie genómov}

	[sekvenovanie DNA a jeho využitie, čítanie (read), párové čítania, kontig, problém najkratšieho spoločného nadslova, de Bruijnove grafy]

\chapter{Zarovnávanie sekvencií}

	[Problém lokálneho a globálneho zarovnania dvoch sekvencií, jeho riešenie pomocou dynamického programovania, skórovacia matica a jej pravdepodobnostný význam, štatistická významnosť (E-value, P-value), heuristické hľadanie lokálnych zarovnaní (BLAST), celogenómové a viacnásobné zarovnania]

\chapter{Hľadanie génov}

	[Gén, exón, intrón, mRNA, zostrih a alternatívny zostrih, kodón, genetický kód, skrytý Markovov model (HMM), jeho stavy, pravdepodobnosti prechodu a emisie, Viterbiho a dopredný algoritmus, použitie HMM na hľadanie génov]

\chapter{Evolúcia a komparatívna genomika}

	[Fylogenetický strom, zakorenenie stromu, metóda maximálnej úspornosti (parsimony), metóda spájania susedov (neighbor joining), metóda maximálnej vierohodnosti (maximum likelihood), Felsensteinov algoritmus, Jukes-Cantorov model substitúcií a zložitejšie substitučné matice, homológ, paralóg, ortológ, detekcia pozitívneho a negatívneho výberu, fylogenetické HMM, test pomerov vierohodností (likelihood ratio test)]

\chapter{Expresia génov, regulácia, motívy}

	[Určovanie génovej expresie pomocou microarray alebo sekvenovaním RNA-seq, hierarchické zhlukovanie, klasifikácia, reprezentácia sekvenčných motívov (väzobné miesta transkripčných faktorov) ako konsenzus, regulárny výraz a PSSM, hľadanie nových motívov v sekvenciách, consensus pattern problem, hľadanie motívu pomocou pravdepodobnostných modelov (EM algoritmus)]

\chapter{Proteíny}

	[Primárna, sekundárna a terciálna štruktúra proteínov, proteínové domény a rodiny, reprezentovanie rodiny pravdepodobnostným profilom a profilovým HMM, protein threading, gene ontology]

\chapter{RNA}

	[Sekundárna štruktúra RNA, pseudouzol a dobre uzátvorkovaná štruktúra, Nussinovovej algoritmus, minimalizácia energie, stochastické bezkontextové gramatiky, kovariančné modely]

\chapter{Populačná genetika}

	[Polymorfizmus, SNP, alela, homozygot, heterozygot, rekombinácia, frekvencia polymorfizmu ako markovovský reťazec, náhodný genetický drift, väzbová nerovnováha (linkage disequilibrium), mapovanie asociácií, LD blok, subpopulácia]
