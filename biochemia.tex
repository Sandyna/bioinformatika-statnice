\chapter[Biochémia]{Biochémia}

\label{biochemia} % id kapitoly pre prikaz ref

\section{Chémia ako logický základ biologického fenoménu}
Prezentácia 1
\subsection{Základné vlastnosti živých systémov}
Zložité a organizované
Bio štruktúry majú funkčný význam
Aktívne zapojené do premien energie
Schopnosť replikácie
Chemický základ
\subsection{Biomolekuly}
HOCN -- schopnosť vytvárať kovalentné väzby cez e-- páry $\rightarrow$ rôzne štruktúry
\subsection{Vlastnosti biomolekúl}
Štruktúrna polarita (napr. 5` $\rightarrow$ 3`)
Informatívnosť (napr. DNA, polypeptidy)
Trojrozmerná štruktúra
Vlastnosti vody
Vysoká hodnota teploty topenia a varu, výparného tepla, povrchového napätia
Polarita $\leftarrow$ Lomená štruktúra
Tvorba vodíkových väzieb
Solvatačné vlastnosti
    Polárne látky $\rightarrow$ vodíkové väzby
    Nepolárne $\rightarrow$ hydrofóbne interakcie
Typy a význam slabých interakcií v biologických štruktúrach
Slabé interakcie udržujú 3D štruktúru a určujú interakcie
    Napr. biomolekulárne rozpoznávanie
    Obmedzené vhodné enviromentálne podmienky
Van der Waalsove
Vodíkové
Iónové
Hydrofóbne
Hydrofóbne interakcie. 
Disperzia lipidov $\rightarrow$ usporiadavajú okolitú H2O
Lipidy sa zoskupujú $\rightarrow$ entropia systému rastie, výhodnejší stav
Micely $\rightarrow$ hydrofóbne konce idú dnu, entropia systému vyššia

2. Aminokyseliny a proteíny.
Všeobecný vzorec AK

klasifikácia AK
    D, L izoméria
    rozdelenie na základe chem vlastností side chain
        náboj
        schopnosť viazať H
        Kyslá/zásaditá
    Nepolárne -- hydrofóbne
    Polárne -- hydrofilné
vzorce AK
Prezentácia 1, str 37, 38
Tvorba disulfidovej väzby
optická aktivita
Schopnosť otáčať rovinu polarizovaného svetla -- napr. Vlnenie fotónu ide zhora dole $\rightarrow$ zľava doprava
Všetky AK okrem glycínu
L a D aminokyseliny 
Prez. 1, str 43, 44
spektroskopické vlastnosti AK
Absorbujú v infrač. oblasti
Trp a tyr, menej Phe v UV
Absorbcia pri 280nm sa používa pri detekcii proteínov

acidobázické vlastnosti AK
Pri nízkom pH je veľa H+, AK stráca čiastočne negatívny náboj a ostane s kladným. 
Pri vysokom pH je veľa OH-- $\rightarrow$ bude mať záporný náboj
Zwitterióny, amfotérny charakter AK, 
Pri neutrálnom pH má oba náboje $\rightarrow$ Zwitterión/Amfión
Vie reagovať s kys. aj zásadami
izoelektrický bod                   
izoelektrický bod -- pH, keď sa AK mení z -- na 0 alebo z + na 0.
    pI = (pKA kyslého + pKA zásaditého) /2. Obyčajne 9 a 2
    pI = average of pKAs of functional groups
štruktúra a vlastnosti peptidovej väzby
Prez 1, str 46
Odchádza/prichádza H2O
N+, O--
medzi jednoduchou a dvojitou
trans
6 atómov v rovine -- planárne usporiadanie
Trojrozmerná štruktúra proteínov -- primárna, sekundárna ($\alpha$--helix, $\beta$--skladaný list, $\beta$--otáčka), terciárna, kvartérna, 
väzby (interakcie) a funkčné skupiny uplatňujúce sa pri jednotlivých štruktúrach
Primárna
poradie AK, kovalentné peptidové väzby
Sekundárna
ako sa skladajú na seba, (základná štruktúra, nie zvyšky), vodíkové väzby medzi CO a NH
$\alpha$--helix (pravotočivý)
    Väzba o 4 zvyšky dopredu
$\beta$--skladaný list
paralelný, antiparalelný
Úplne rozvinutý reťazec
Väzby aj medzi rozdielnymi reťazcami
$\beta$--otáčka
    Zmena smeru peptidového reťazcu
    Väzba o 3 zvyšky ďalej
prolín, glycín
Terciárna
Priestorová štruktúra, interakcie vzdialených skupín, ako sa folds skladajú na seba, vodíkové väzby, Van der Waals, hydrofóbny obal, disulfidový mostík medzi bočnými reťazcami
Daná primárnou štruktúrou
kvartérna
    medzi rôznymi polypeptidmi
    podjednotky sa skladajú do mérov -- diméry, tetraméry, multiméry $\rightarrow$ počty polypeptidových reťazcov
    Homo/hetero multimérne -- rovnaké/rôzne reťazce
TODO: Kedy sa rozpadajú? Zmena pH, teplota, atď?
Cysteín -- disulfidový mostík

Rozdelenie proteínov podľa štruktúry a rozpustnosti (fibrilárne, globulárne, membránové proteíny)     
Fibrilárne
pevné, reťazce väčšinou paralelné s jednou osou
nerozpustné, štruktúrna funkcia
keratíny, kolagén, fibroín
Prez 1, str 73
Globulárne
hydrofilné von, hydrofóbne dnu
Flexibilné časti, štruktúry nie sú statické (PARTAAY)
mioglobín, cytochróm c, lyzozým, ribonukleáza
Membránové
    bakteriorodospín
Biologická funkcia proteínov, natívna konformácia, denaturácia, renaturácia. 
Enzýmová katalýza
Transportná, zásobná -- hemoglobín(O2), sérumalbumín (MK), Ovalbumín, Kazeín (N), Ferritín (Fe)
Koordinovaný pohyb -- Aktín, myozín
mechanická podpora -- kolagén, keratín
Imunita
nervové impulzy
regulácia rastu, diferenciácia
                   
natívna konformácia -- správne zložený proteín. Aktívna forma. Chyby na hociktorej úrovni vedú ku chorobám
Denaturácia
unfolded, neaktívny
pH, teplota, chemikálie, org. rozpúšťadlá, detergenty, močovina, enzýmy
Neovplyvňuje primárnu štruktúru.
vratná/nevratná
Renaturácia
    Nie vždy sa poskladá správne
Chaperone -- proteín, čo skladá správne proteíny
Prirodzene neusporiadané proteíny $\rightarrow$ viac funkcií, nemávajú hydrofóbne jadro

3. Sacharidy
Rozdelenie sacharidov, aldózy, ketózy
aldózy -- O na začiatku
ketózy -- O v strede
mono, oligo, poly
lineárne, rozvetvené

Vzorce (lineárne -- Fischerove, cyklické -- Haworthove): glukóza, manóza, galaktóza, ribóza. Pojmy: konfigurácia, konformácia, enantiomér, epimér, diastereomér, poloacetál, poloketál, mutarotácia, $\alpha$--, $\beta$--anoméry. Vznik glykozidovej väzby. Deriváty sacharidov (kyseliny, alkoholy, deoxysacharidy -- deoxyribóza, estery sacharidov, aminosacharidy -- glukozamín, acetály, ketály, glykozidy). Disacharidy (redukujúce, neredukujúce disacharidy, príklady -- laktóza, sacharóza, trehalóza). Štruktúrne polysacharidy -- celulóza, chitín
(väzby, štruktúra). Zásobné polysacharidy -- škrob, glykogén (väzby, štruktúra).
Heteropolysacharidy -- peptidoglykán, hyaluronát, proteoglykány (základná charakteristika). Sacharidy ako informačné molekuly. Lektíny

4. Lipidy a biologické membrány. Funkcie lipidov. Štruktúra a vlastnosti mastných kyselín (kyselina palmitová, steárová, olejová, linolová, linolénová). Triacylglyceroly (tuky, oleje), glycerofosfolipidy (fosfatidyletanolamín, fosfatidylcholín, fosfatidylserín, fosfatidylglycerol, fosfatidylinozitol, kardiolipín), sfingolipidy (sfingomyelíny, cerebrozidy, ceramidy, gangliozidy), vosky, cholesterol -- štruktúra a funkcia. Amfipatický charakter niektorých lipidov, agregované formy lipidov -- micely,
dvojvrstvy. Princíp samovoľného vzniku lipidových agregátov. Biomembrány, membránové proteíny, model tekutej mozaiky. Úloha cholesterolu pri ovplyvňovaní fluidity membrán. Transport cez membrány (pasívny, aktívny). Na+/K+ pumpa.

5. Enzýmy. Význam enzýmovej katalýzy. Pojmy -- holoenzým, apoenzým, kofaktor, koenzým, prostetická skupina. Klasifikácia enzýmov. Aktívne miesto, špecificita enzýmov. Jednotka enzýmovej aktivity -- katal. Mechanizmus účinku enzýmov -- teória komplementarity, teória indukovaného prispôsobenia. Termodynamické hľadisko priebehu enzymaticky katalyzovaných reakcií, aktivačná energia, prechodný stav. Kinetické hľadisko priebehu enzymaticky katalyzovaných reakcií, faktory ovplyvňujúce rýchlosť enzýmovej reakcie, Michaelis -- Mentenovej
rovnica, parametre Km a Vmax; inhibícia enzýmov -- ireverzibilná, reverzibilná -- kompetetívna, nekompetetívna. Regulácia enzýmov -- alosterickou modifikáciou, kovalentnou modifikáciou, regulačnými proteínmi, proteolytickým štiepením (zymogény).

6. Základy metabolizmu. Zdroj a premeny energie v biosfére. I. a II. zákon termodynamický. Chemická energia -- entalpia, voľná (Gibbsova) energia, entropia. Endergonické, exergonické reakcie. Podmienka samovoľnosti priebehu chemických dejov. Význam prenášačov energie, úloha, vznik (substrátová fosforylácia, oxidačná fosforylácia, fotofosforylácia) a premeny ATP. Katabolické a anabolické metabolické dráhy, ich význam. Energetické vzťahy medzi katabolickými a anabolickými dráhami. Oxidácia biomolekúl.

7. Metabolizmus sacharidov. Glukóza ako zdroj metabolickej energie. Glykolýza -- význam, lokalizácia, 2 fázy glykolýzy, jednotlivé reakcie, medziprodukty a enzýmy glykolýzy. Spotreba a vznik ATP počas glykolýzy, substrátová fosforylácia. Osud pyruvátu a regenerácia NAD+, anaeróbne -- mliečne kvasenie, alkoholové kvasenie, aeróbne -- v dýchacom reťazci. Glukoneogenéza -- význam, substráty, tri unikátne glukoneogenetické kroky (4 enzýmy), lokalizácia. 

Coriho cyklus, prenos laktátu zo svalu do pečene, vznik glukózy z laktátu procesom
glukoneogenézy. Pentózová dráha: význam, východisková zlúčenina, vznik NADPH,
ribulóza--5--fosfátu, reakcie katalyzované dehydrogenázami, izomerázou, epimerázou,
transaldolázami, transketolázami.

8. Citrátový cyklus. Glyoxylátový cyklus. Vznik acetyl--koenzýmu A z kyseliny pyrohroznovej.
Citrátový cyklus -- zdroj energie a biosyntetických prekurzorov, bunková lokalizácia cyklu. Reakcie
citrátového cyklu, jednotlivé medziprodukty a enzýmy. Vznik redukovaných koenzýmov. Tvorba
GTP -- substrátová fosforylácia. Amfibolický charakter citrátového cyklu, anaplerotické reakcie
(pyruvátkarboxyláza). Glyoxylátový cyklus -- význam pre rastliny a baktérie, lokalizácia (spolupráca
glyoxyzómov a mitochondrií), enzýmy.

9. Oxidačná fosforylácia. Štruktúra a funkcia mitochondrií. Zloženie a funkcia dýchacieho reťazca,
prenášače elektrónov -- cytochrómy, bielkoviny s nehemovo viazaným železom, ubichinón,
flavoproteíny. Zdroj elektrónov vstupujúcich do dýchacieho reťazca. Prenos elektrónov v dýchacom
reťazci (komplexy I, II, III, IV, cyt c, ubichinón). Vznik protónového gradientu. Využitie protónového
gradientu na syntézu ATP, enzým ATP--syntáza. Chemiosmotická teória. Ďalšie možnosti využitia
protónového gradientu -- termogenéza, pohyb baktérií, transport metabolitov.
10. Fotosyntéza. Fotofosforylácia ako súčasť fotosyntézy. Štruktúra a funkcia chloroplastov. Pigmenty a ich úloha v procese fotosyntézy. Fotochemické reakčné centrum a deje, ktoré v ňom prebiehajú.
Prenos elektrónov fotosystémami I a II. Necyklická a cyklická fotofosforylácia. Fotolýza vody. Vznik NADPH a ATP. Spoločné a rozdielne znaky fotofosforylácie a oxidačnej fosforylácie. Syntézasacharidov počas fotosyntézy. Tri štádiá asimilácie CO2. Základné reakcie a funkcia Calvinovhocyklu.

11. Metabolizmus lipidov. Mastné kyseliny ako zdroj metabolickej energie. Trávenie tukov -- význam žlčových kyselín, enzýmov lipáz; chylomikrónov. Osud mastných kyselín vo svaloch a v tukovom tkanive. Uvoľnenie mastných kyselín z tukového tkaniva a ich prenos do tkanív (funkcia sérumalbumínu). $\beta$--oxidácia mastných kyselín -- lokalizácia v bunke, prenos mastných kyselín do mitochondrií (funkcia karnitínu). Reakcie $\beta$--oxidácie -- dehydrogenácia, hydratácia, dehydrogenácia, štiepenie, vznik acetyl--kaoenzýmu A. Osud acetyl--koenzýmu A -- vstup do citrátového cyklu; vznik
ketolátok, ich význam. Biosyntéza mastných kyselín -- porovnanie s $\beta$--oxidáciou, východiskové zlúčeniny, reakcie kondenzácia, redukcia, dehydratácia, redukcia. Zdroje NADPH. Transport triacylglycerolov a cholesterolu u ľudí, lipoproteíny.

12. Degradácia aminokyselín. Aminokyseliny ako zdroj metabolickej energie. Odbúranie aminokyselín -- odstránenie aminoskupiny transamináciou a deamináciou (enzýmy transaminázy, glutamátdehydrogenáza). Význam glutamínu pri odbúraní AK (enzýmy glutamínsyntetáza, glutamináza). Formy vylučovania aminoskupiny u rôznych stavovcov. Močovinový cyklus -- orgánová a bunková lokalizácia, význam. Osud uhlíkovej kostry aminokyselín, glukogénne, ketogénne aminokyseliny.



